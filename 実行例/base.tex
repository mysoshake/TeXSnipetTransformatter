% !TEX TS-program = uplatex
% !TEX encoding = UTF-8 Unicode
%%% 設定開始 %%%
% \documentclass{ltjsarticle}
\documentclass[report]{jlreq}% loads luatexja.sty internally

%%%% 基本
% スタイル設定
	\usepackage[utf8]{inputenc} % Unicode使用可能
	% \usepackage{csquotes} % 謎に警告が出るから
	% \usepackage[ngerman]{babel} % ドイツ語にする
	\usepackage{luatexja-fontspec} % フォント設定 
	\setmainfont[Ligatures=TeX]{Times New Roman} % フォント設定
	\setmainjfont[BoldFont=MS Gothic]{MS Mincho} % フォント設定
	\usepackage[top=20truemm,bottom=20truemm,left=25truemm,right=20truemm]{geometry} % ページ設定
	
	% 機能系
	\usepackage{amsmath} % 数式サポート
	\usepackage{amssymb} % 数式記号サポート
	\usepackage{pxrubrica}
	\usepackage{bm} % 太字 \bm 使用可能
	\usepackage{tcolorbox} % 定理の枠使用可能
	\tcbuselibrary{breakable, skins, theorems} % 定理の枠使用可能
	\usepackage{hyperref} % ハイパーリンク使用可能
	\usepackage{cleveref} % 表\ref{hoge}をref{hoge}に ※ハイパーリンクの後に読み込まないとだめらしい!
	\usepackage{graphicx} % 画像貼り付け使用可能
	\usepackage{svg} % svg読み込み可能
	\usepackage{here} % 画像表示位置
	\usepackage{listings, jvlisting} % プログラムソース表示可能
	\usepackage{enumitem} % 列挙(箇条書き)使用可能
	\usepackage{biblatex} % BIBLaTeX使用可能
	% sorting == n[name], t[title], y[year] v[volume], a[alphabetic], Xd(descending), none[Do not sort at all]
	\usepackage{url} % url 使用可能
	%\usepackage{caption} %タイトル使用可能
	%\usepackage{subcaption} % 
	\usepackage{multirow} % 表の行結合可能
	\usepackage{physics} % 
	\usepackage{array} % 表の幅と中心よせなど
	\usepackage{hhline} % 水平線を引いたとき縦が途切れない
	\usepackage{lltjext} % 縦書き使用可能
	\usepackage{tikz} % TikZ使用可能
	\usepackage{tikzsymbols} % TikZの文字使用可能
	\usetikzlibrary{calc, intersections, angles, arrows.meta} % Tikz 拡張ライブラリ
	\usepackage{circuitikz} % 論理回路使用可能
	\usepackage{pdfpages} % PDF読み込み可能
	\usepackage{fancyhdr}
	
	
	% よく変更するもの
	\bibliography{library}	%参考文献の取り込み(library.bibというファイルが存在する場合)
	
	\pagestyle{fancy} % ページ番号を消すなど {empty, plain, headings, myheadings, fancy}
	\fancyhead[L]{}  % = \lhead{ヘッダー・左}
	\fancyhead[C]{}  % = \chead{ヘッダー・中央}
	\fancyhead[R]{\thepage}  % = \rhead{ヘッダー・右}

	\fancyfoot[L]{}  % = \lfoot{フッター・左}
	\fancyfoot[C]{}  % = \cfoot{フッター・中央}
	\fancyfoot[R]{}  % = \rfoot{フッター・右}
	\setlength{\headheight}{17.0pt}

	\renewcommand{\headrulewidth}{0mm}
	\renewcommand{\footrulewidth}{0mm}
	
	\lstset%
	{%
		basicstyle={\ttfamily},
		identifierstyle={\small},
		commentstyle={\smallitshape},
		keywordstyle={\small\bfseries},
		ndkeywordstyle={\small},
		stringstyle={\small\ttfamily},
		frame={tb},
		breaklines=true,
		columns=[l]{fullflexible},
		numbers=left,
		xrightmargin=0\zw,
		xleftmargin=3\zw,
		numberstyle={\scriptsize},
		stepnumber=1,
		numbersep=1\zw,
		lineskip=-0.5ex,
	}%
	\renewcommand{\lstlistingname}{リスト}

	% Clever Reference  \crefname{環境名}{単数形}{複数形}
	%\crefname{equation}{式}{式}
	%\crefname{figure}{図}{図}
	%\crefname{table}{表}{表}
	%\crefname{lstlisting}{リスト}{リスト}
	
	% カスタマイズ
	\renewcommand{\thesection}{\arabic{section}}
	\newcommand{\dirow}[2]{\multirow{2}{#1}{#2}} % セルの結合
	\newcommand{\trirow}[2]{\multirow{3}{#1}{#2}} % セルの結合
	\newcommand{\tgb}[1]{\textsf{\textgt{#1}}} % 太字
	
	\ctikzset
	{%
		logic ports=ieee,
		logic ports/scale=1.0,
	}

